\documentclass{article}
\title{Proposal for a paper}
\author{Sean et al.}
\begin{document}
\maketitle

\section{Overview}

The paper will concern an adversarial communications game
(i.e.\ electronic warfare) between a blue side, with one immobile
headquarters, $n_a$ immobile assets, $n_c$ mobile comms units (UAVs),
and a red side, with $n_j$ mobile jammers (also UAVs).

At each time step, each blue unit determines whether it can detect a
radio signal from each of the other blue units, or, alternatively,
whether any of these signals fails to get through (because it is
either jammed or out of range).  Using this set of $(n_c + n_a +
1)(n_c + n_a)$ binary results (for each time step), I have shown that
blue can infer the locations of the red jammers quite accurately,
assuming each blue unit has knowledge of its own position and of the
positions of all other blue units.

In addition to attempting to receive signals from all other blue
units, each blue unit attempts to send a signal to each other blue
unit, so that all can perform the same tests.  These signals broadcast
(hidden or ``encrypted'' from red units) any information that another
blue unit could conceivably make use of (e.g.\ the sender's position,
or its successes or failures to communicate with other blue units).
Note that the ability to communicate is not always bidirectional even
when initial signal strengths are equal.  This asymmetry is due to the
asymmetrical placement of jammers.

After blue infers the positions of the red jammers it then moves its
comms to counter the jamming so that as many units as
possible---especially the assets---stay in communication with the
headquarters.  Specifically, the goal of the blue side is to maximize
cumulative reward, where the reward on a given time step is 10 times
the number of assets in (direct or indirect) communication with the
headquarters, plus one times the number of comms in (direct or
indirect) communication with the headquarters.  Indirect communication
with headquarters is assessed by analyzing the connected
component---containing headquarters---of the graph of the
communication links between units.

As mentioned above, I have coded a proof-of-concept for the blue side,
where, on each time step, each red jammer moves one or zero grid units
in each xy-component direction and comms teleport to a new random
location on the grid.  The model for red transition is known
accurately to the blue side.  I have found that inference of positions
of red side jammers works quite well, with delay upon jammer moves.

I haven't thought through the problem of moving red side jammers to
optimally block the blue side.  How will they collect information on
the locations of their enemy?  How will they infer enemy positions?
What reward function will they use?  The problem of finding blue units
is very similar to the problem of finding the red jammers but it does
not seem completely symmetric.  Still, I need to articulate in what
ways is it different.  One difference is that jammers jam on a whole
swath of the frequency band whereas comms attempt to communicate at
just one particular narrow frequency.  This makes jammers easy to find
and comms hard to find.
\end{document}
